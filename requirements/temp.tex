\subsection{RQ1: Detector Effectiveness}
This research question aims to evaluate the effectiveness of our Factory Contract Detector through
metrics including precision, recall, and execution efficiency. Validating the detector's
effectiveness is crucial as it ensures the accuracy and reliability of our factory contract
identification, which forms the foundation for all subsequent analyses (RQ2-RQ5).

\textbf{Ground-Truth Dataset Construction.} We construct a ground-truth dataset to evaluate
detector effectiveness through two complementary approaches. (i) For factory contracts, we
utilize Google BigQuery's Ethereum \textit{traces} table~\cite{bigquery-ethereum-traces} to identify
contracts that executed CREATE or CREATE2 operations in non-constructor contexts with successful
status, ensuring definitive factory classification through on-chain execution records. (ii) For non-factory
contracts, we obtain verified smart contracts from Etherscan~\cite{etherscan-verified-contracts}
and filter those whose source code contains no ``new'', ``create'', or ``create2'' keywords, guaranteeing
non-factory classification. This methodology yields 548 factory contracts with unique bytecode
and 2359 non-factory contracts with unique bytecode for evaluation.

\textbf{Precision and Recall.} To evaluate the performance of our factory detector, we define
the following classification categories:
\begin{itemize}[leftmargin=0.4cm,topsep=0.1cm]
	\item \textbf{TP (True Positive):} The current contract is a factory contract, and the factory
	detector correctly identifies it as a factory contract.

	\item \textbf{FP (False Positive):} The current contract is not a factory contract, but the factory
	detector incorrectly classifies it as a factory contract.

	\item \textbf{FN (False Negative):} The current contract is a factory contract, but the factory
	detector fails to identify it as such.

	\item \textbf{TN (True Negative):} The current contract is not a factory contract, and the factory
	detector correctly identifies it as a non-factory contract.
\end{itemize}
We calculate the precision and recall metrics to assess our detector's effectiveness. Precision
= TP/(TP+FP) and Recall = TP/(TP+FN).

Our evaluation on the complete dataset of 2,907 contracts yields a precision of \textbf{99.21\%} and a recall of \textbf{91.61\%}. Specifically, out of 548 factory contracts, our detector correctly identifies 502 (TP = 502) while missing 46 contracts (FN = 46). Among 2,359 non-factory contracts, our detector correctly classifies 2,355 (TN = 2,355) but incorrectly flags 4 as factories (FP = 4).

\textbf{False Negative Analysis.} Our analysis of the 46 false negatives reveals that all misclassified cases are attributable to proxy-based factory patterns. These contracts delegate their factory functionality to implementation contracts through proxy mechanisms, where the actual CREATE operations occur in the delegated bytecode rather than the proxy contract's own bytecode. Among these cases, 13 contracts (20.6\%) utilize CREATE2-generated vanity addresses with leading zeros, indicating sophisticated deployment strategies. The execution traces confirm that these contracts successfully performed factory operations (ranging from 640 to 6,902,508 CREATE operations), but our static bytecode analysis cannot detect the delegated factory logic. This limitation represents the primary constraint of our detector: inability to analyze proxy-based factory architectures where the factory functionality is abstracted through delegation patterns.

\textbf{False Positive Analysis.} Our analysis of the 4 false positives demonstrates exceptionally high precision in factory detection. These misclassifications result from contracts containing CREATE/CREATE2 bytecode patterns that are structurally similar to factory operations but serve different purposes. Specifically, these contracts include CREATE-related bytecode sequences within initialization code or as part of complex deployment mechanisms, leading our static analysis to incorrectly interpret them as factory capabilities. The extremely low false positive rate (0.17\% of non-factory contracts) validates the conservative design of our detector.

Our Factory Detector can effectively identify factory contracts through static bytecode analysis with high precision and recall, ensuring the reliability of factory contract classification for subsequent experimental analyses. The primary limitation—proxy-based factory detection—represents a well-defined boundary that enables confident interpretation of our results.

\textbf{Execution performance.} Our factory detector demonstrates excellent computational efficiency across the entire evaluation dataset. The execution time analysis reveals that 25\% of contracts can be analyzed within 1.7 milliseconds, while 5\% of contracts complete analysis in 4.7 milliseconds or less. Furthermore, 75\% of all contracts achieve complete factory detection within 24.2 milliseconds, demonstrating the scalability of our approach for large-scale analysis.

